\documentclass{amsart}
\usepackage{amsfonts}
\usepackage{amsmath}
\usepackage{amsthm}
\usepackage{amssymb}
\usepackage{mathtools}
\usepackage{enumerate}
\usepackage{graphicx}
\usepackage{dsfont}
\usepackage{bbm}
\usepackage{tikz}
\usepackage{xifthen}
\usepackage{cancel}

\setlength{\textwidth}{\paperwidth}
\addtolength{\textwidth}{-2in}
\calclayout


\newcommand{\NN}{\mathbb{N}}
\newcommand{\RR}{\mathbb{R}}
\newcommand{\HH}{\mathbb{H}}
\newcommand{\QQ}{\mathbb{Q}}
\newcommand{\ZZ}{\mathbb{Z}}
\newcommand{\Zn}[1]{\mathbb{Z} / #1 \mathbb{Z}}
\newcommand{\CC}{\mathbb{C}}
\newcommand{\FF}{\mathbb{F}}
\newcommand{\PP}{\mathbb{P}}
\newcommand{\dd}[2][ ]{\frac{\partial #1}{\partial #2}} %partial derivative
\newcommand{\dsq}[3][ ]{\frac{\partial^2 #1}             %Second partial
{\ifthenelse { \equal {#2} {#3} }{\partial #2^2}{\partial #2 \partial #3}}}
\newcommand{\defarr}					    %definition iff arrow
{\overset{\textrm{def}}{\Longleftrightarrow}}
\newcommand{\defeq}						    %definition equality sign
{\overset{\textrm{def}}{=}}
\newcommand{\argeq}[1]						%definition equality sign
{\overset{\textrm{#1}}{=}}

\DeclareMathOperator{\lcm}{lcm}
\DeclareMathOperator{\orb}{orb}
\DeclareMathOperator{\im}{im}
\DeclareMathOperator{\supp}{supp}
\DeclareMathOperator{\stab}{Stab}
\DeclareMathOperator{\sgn}{sign}
\DeclareMathOperator{\spn}{span}
\DeclareMathOperator{\tr}{tr}

\newtheorem{thm}{Theorem}[section]
\newtheorem{lemma}[thm]{Lemma}
\newtheorem{fact}[thm]{Fact}

\theoremstyle{definition}
\newtheorem{defn}[thm]{Definition}
\theoremstyle{remark}
\newtheorem*{rmk}{Remark}
\newtheorem*{ex}{Example}
%==============================================================================
% MATH 494 Collaborative Notes
% This is a collaborative notesheet that consists of notes from each day in Math
% 494, transcribed into Tex format for convenience and security from the
% Notebook Thief. General format:
% Contribute by adding a new section, titled "Month Day, Year"
% This is done in the format \section{Date}
% Tex notes using the numbering system used by Tasho himself. I've set up the
% counters on the theorems, definitions, and "facts" to run in tandem with th
% section that the content is in.
% Make sure to properly wrap all proofs, facts, and definitions using the above
% environments.
% Enjoy!
% Current Collaborators:
% 1. Pranav
% 2.
% 3.
% 4.
%==============================================================================
\begin{document}
\title{Math 494: Honors Algebra II}
\maketitle
\section{January 1, 2017}
\noindent \textbf{Rings}
\begin{defn} \hspace{0.5cm}
    \begin{enumerate}[a)]
    \item A \textbf{ring} is a tuple $(R, +, \cdot, 0)$ where:
    \begin{itemize}
        \item $R$ is a set
        \item $0 \in R$
        \item $+,\cdot: R \times R \rightarrow R$, $\quad$  $(a,b) \mapsto a + b, a \cdot b$
    \end{itemize}
    subject to:
    \begin{itemize}
        \item $(R, +, 0)$ is an abelian group
        \item $(a \cdot b) \cdot c = a \cdot (b \cdot c)$
        \item $(a + b) \cdot c = a \cdot c + b \cdot c$
        \item $a \cdot (b + c) = a \cdot b + a \cdot c$
    \end{itemize}
    \item A \textbf{ring with unity} is a tuple $(R, +, \cdot, 0, 1)$, where
    $(R,+,\cdot,0)$ is a ring, and $1 \in R$ is subject to $1 \cdot a = a \cdot 1 = a$
    for all $a \in R$.
    \item A ring $(R, +, \cdot, 0)$ is called \textbf{commutative} if $ab = ba$ for all
    $a, b \in R$.
    \item A \textbf{field} is a commutative ring with unity $(R,+,\cdot,0,1)$ such
    that $(R \backslash \{0\}, \cdot, 1)$ is a group.
    \end{enumerate}
\end{defn}
\begin{rmk} \hspace{0.5cm}
    \begin{itemize}
        \item We don't really need to include 0,1 in notation: they are unique
        if they exist
        \item There is a notion of a \textbf{skew field}: ring with unity
        $(R,+,\cdot,0,1)$ such that $(R \backslash \{0\}, \cdot , 1)$ is a group.
        (This drops the commutative condition from the definition of a field).
        \item In French: \textit{corps} is a skew field, and \textit{corps commutatif} is a field.
    \end{itemize}
\end{rmk}
\begin{fact}\label{fact:0prod}
 Let $R$ be a ring. For all $a \in R$, $0 \cdot a = 0$.
\end{fact}
\begin{proof}
    $(0 \cdot a) = (0 + 0) \cdot a = 0 \cdot a + 0 \cdot a \Rightarrow 0 = 0 \cdot a$
\end{proof}
\begin{ex} \hspace{0.5cm}
    \begin{itemize}
        \item $\ZZ$ is a ring, commutative, with unity
        \item $\QQ, \RR, \CC$ are fields
        \item $\HH = \{a + bi + cj + dk \mid a,b,c,d \in \RR \}$ where $i^2 = j^2 = k^2 = ijk = -1$ are
        called the \textbf{Hamiltonian Quaternions} and are a skew-field
        \item $\mathcal{C}^{\infty}_{C}(\RR) = $ functions on $\RR$ with compact
        support \\
        ($\supp(f) = \overline{\{x \in \RR \mid f(x) \neq 0\}}$) is a
        commutative ring without unity
        \item $R = \{\star\}, 0 = 1 = \star$ is the \textbf{zero ring}.
    \end{itemize}
\end{ex}
\begin{fact}
    If $(R,+,\cdot,0,1)$ is a ring with unity and $0 = 1$, then $R$ is the zero ring.
\end{fact}
\begin{proof}
    Take $a \in R$. Then $a = a \cdot 1 = a \cdot 0 = 0$ by Fact \ref{fact:0prod}.
\end{proof}
\noindent \underline{Convention}: Unless otherwise noted, ring will refer to
a commutative ring with 1.
\begin{defn}
    Let $R$ be a ring. Its \textbf{group of units} is
    $$
    R^\times = \{a \in R \mid \exists \, b \in R: ab = 1\}
    $$
\end{defn}
\begin{fact} \hspace{0.5cm}
    \begin{itemize}
        \item For $a \in R^\times$, there is a unique $b \in R$ such that $ab = 1$.
        Write $b = a^{-1}$.
        \item For $a,b \in R^\times$, $a \cdot b \in R^\times$.
    \end{itemize}
\end{fact}
\begin{proof} \hspace{0.5cm}
    \begin{itemize}
        \item Given $b, b^\prime$, we have $b = b \cdot 1 = b(ab^\prime) = (ba)b^\prime = 1 \cdot b^\prime
         = b^\prime$.
         \item $(a \cdot b) \cdot (b^{-1} \cdot a^{-1}) = 1$
    \end{itemize}
\end{proof}
\begin{ex}
    $\RR^\times = \RR \backslash \{0\}$, $\ZZ^\times = \{1, -1\}$
\end{ex}
\begin{defn}
    Let $R, S$ be rings. A \textbf{morphism} $\phi:R \rightarrow S$ is a map of
    sets $\varphi:R \rightarrow S$ satisfying
    \begin{itemize}
        \item $\varphi(a + b) = \varphi(a) + \varphi(b)$
        \item $\varphi(a \cdot b) = \varphi(a) \cdot \varphi(b)$
        \item $\varphi(1) = 1$
    \end{itemize}
\end{defn}
\begin{ex}
    $\varphi:\ZZ \rightarrow \ZZ$ $u \mapsto 0$ is \underline{not} a morphism of
    rings with 1. (it is a morphism of general rings).
\end{ex}
\begin{fact}
    For any ring $R$ there is a unique morphism $\varphi:\ZZ \rightarrow R$. Given
    $z \in \ZZ$, we write $z_{R}$, or simply $z$ for its image under $\varphi$.
\end{fact}
\begin{ex}
    $5 \in \ZZ$, $5_{\QQ} \in \QQ$ usual number $5$. $5_{\Zn 2} = 1_{\Zn 2}$
\end{ex}
\begin{defn}
    Let $R$ be a ring. A subset $I \subset R$ is called an \textbf{ideal} if
    \begin{itemize}
        \item $I$ is a subgroup of $(R, +, 0)$
        \item $a \cdot f \in I$ for all $a \in R, f \in I$.
    \end{itemize}
\end{defn}
\begin{defn}
    Let $R$ be a ring. A subset $S \subset R$ is called a subring if
    \begin{itemize}
        \item $S$ is a subgroup of $(R, +, 0)$
        \item $a \cdot b \in S$ for all $a \cdot b \in S$.
        \item $1 \in S$.
    \end{itemize}
\end{defn}
\begin{rmk}\hspace{0.5cm}
    \begin{itemize}
        \item The only subset that is both a subring and an ideal is $R$ itself.
        (reason: if $1 \in I$, then $a \cdot 1 \in I$ for all $a \in R$, meaning $I = R$)
        \item $I = \{0\}, I = R$ are always ideals.
        \item In rings without unity, the 2 notions align closer: ideal becomes a special
        case of substring as $1 \in S$ condition is dropped.
    \end{itemize}
\end{rmk}
\begin{ex} \hspace{0.5cm}
    \begin{itemize}
        \item Every subgroup of $(\ZZ, +, 0)$ is an ideal of $\ZZ$.
        \item If $F$ is a field, then $\{0\}, R$ are the only ideals
        \item Let $R = \mathcal{C}_C(\RR), S \in R$ subset.
        $$
        I = \{f \in \mathcal{C}_C(\RR) \mid f \mid_{S} = 0 \}
        $$
        is an ideal
    \end{itemize}
\end{ex}
\begin{defn}
    An ideal $I \in R$ is called \textbf{principal} if $I = \{a \cdot r \mid r \in R\}$
    for some $a \in R$. Then $a$ is called a \textbf{generator}.
\end{defn}
\begin{defn}
    Let $a_1, a_2, \dots a_n \in R$. An \textbf{ideal generated by} $a_1, \dots a_n$ is
    $$
    (a_1, \dots a_n) = \{a_1r_1 + \dots + a_nr_n \mid r_i \in R\}
    $$
\end{defn}
\begin{fact}
    Given ideals $I, J \subset R$ we have
    \begin{itemize}
        \item $I \cap J$ is an ideal
        \item $I + J = \{a + b \mid a \in I, b \in J \}$ is an ideal
        \item $I \cdot J = \left\{\sum\limits_{i = 1}^{n}a_ib_i \mid a_i \in I, b_i \in J \right\}$ is an ideal
    \end{itemize}
\end{fact}
\end{document}
